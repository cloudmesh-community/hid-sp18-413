\section{Puppet}

{\bf you are not using citations, no urls allowed in abstracts}

Puppet~\cite{hid-sp18-413-puppet} is a open source software
configuration and automation tool. It is written in C++ and
Clojure. Puppet is a declarative language and uses domain specific
language for configuration. Puppet uses facter to gather information
about the system and user defines the desired state. Puppet does not
use sequential programming where order of execution is key but uses
graphical representation to represent the order of
execution. Resources are defined in manifests written in Domain
specific language. These manifests are complied into catalogue on
puppet master and supplied to puppet clients. These catalogues are
only applied if actual and desired states are different.
``\href{https://en.wikipedia.org/wiki/Kubernetes}{Kubernetes} is new
cluster manager from google'' and puppet makes it easy to manage the
kubernetes resources. Puppet is declarative, modular, has code testing
features and therefore managing kubernetes with it is easier.
