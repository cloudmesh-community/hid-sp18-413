\title{Puppet}


\author{Anubhav Lavania}
\affiliation{%
  \institution{Indiana University}
  \streetaddress{Smith Research Center}
  \city{Bloomington} 
  \state{IN} 
  \postcode{47408}
  \country{USA}}
\email{alavania@iu.edu}


% The default list of authors is too long for headers}
\renewcommand{\shortauthors}{G. v. Laszewski}


\begin{abstract}
  This paper provides a brief summary of Puppet and briefly discusses its
  implementation with Kubernetes.Puppet is a open source software configuration
  and automation tool. It is written in C++ and Clojure. Puppet is a declarative
  language and uses domain specific language for configuration. Puppet uses
  facter to gather information about the system and user defines the desired
  state. Puppet does not use sequential programming where order of execution is
  key but uses graphical representation to represent the order of
  execution. Resources are defined in manifests written in Domain specific
  language. These manifests are complied into catalogue on puppet master and
  supplied to puppet clients. These catalogues are only applied if actual and
  desired states are different.Kubernetes is cluster manager from googl and
  puppet makes it easy to manage the kubernetes resources. Puppet is
  declarative, modular, has code testing features and therefore managing
  kubernetes with it is easier.
\end{abstract}

\keywords{hid-sp18-413, puppet, Kubernetes}


\maketitle



\section{Introduction}

Puppet is configuration management and automation software that works in a
standalone or client server mode. Puppet is made up of two parts, puppet
language and puppet platform. In contrast to procedural language, desired client
configuration is described in Puppet Domain Specific language and it is applied
via catalogues. Puppet master compiles the catalogue which is applied on servers
running puppet agents. Puppet employs idempotency meaning it does not apply anychanges if system is already in desired state.
~\cite{editor00}.

\section{Puppet Architecture}
Puppet works in standalone and master agent configuration. Puppet works in two
steps, first by compiling a catalogue and second by applying it. When puppet is
working in master agent configuration, agent requests the catalogue, master
complies and it is then applied on the agent. On the agent node puppet agent
checks the catlogue for the resources and if any is not in desired state,
changes are applied. Communication between puppet agent and master is https via
ssl. Agents ssl certificates must be verified by master for them to
communicate~\cite{hid-sp18-413-puppet-architecture}. \par
Puppet can also work in standalone mode. In standalone mode puppet apply is used
to apply the catalogues. In smaller organisations which cannot have dedicated
puppet masters and puppetdb, standalone configuration works just fine although  standalone mode comes with its own sets of prolems and
is usually not recommended. Standalone mode is decentralized and configuration
change on multiple devices need to be run individually and then synced with
other devices. Cpu and memory utilization will be high. If running puppetdb it
is much easier to run reports and query resources, with standalone agents, it is
not possible. 

\section{Puppet Language}

Puppet is a declarative language. User defines the end state or desired state of
system and is not bothered by how that state is reached for example to install
open-shh different operating systems will use different package manager like
apt-get, yum, pkgadd or msi and then define the steps of how installation should
proceed.In puppet you declare that for example package open-ssh should be
prsesent and same code will work on various operating systems. This is achived
by defining resources using types, attributes, values and metaparameters~\cite{hid-sp18-413-puppet-language}.  Resource describes the resource to
me managed and it could be package, user, service etc.Puppet groups similar
types of resources into types. Users fall into one type, files into another and
services into another. Title of resource
distinguishes a resource from other resources. These must be unique for example openssh.  Attribute
describe the desired state of resource. For example ensure, present, owner,
file.  Attributes have values which define the desired state of attribute for
example, mode, present, mode.  Metaparametres like before and require (Define
dependency relationships) File resource defines source to get config to be
deployed on agents.
Variables in puppet are infact constants and can be assigned any value from
multiple data types. Varables in local scope are available locally and to child
scopes and any value assigned from outer scopes is overrulled. Out of scope
varables are assigned by using qualified name of its scope.Variables in puppet
are immutable meaning value of varaiable cannot be changed in its scope. But
same variable name can have a different value in another scope.


\section{Puppet Modules}

Similar to any other programming language Puppet makes use of modules
to wtite reusable and modular code~\cite{hid-sp18-413-puppet-modules}.Puppets
module are collection of classes, resource types, files and templates. Puppet
has a rich library of existing modules that are freely available to use. In
addition to existing modules users can create their own modules. Puppet follows a directory
structure for defining modules and they must be ordered. Each puppet
module must have a valid name and must be located in module path. When
looking for modules puppet searches module path and makes all
classes and resources available for all modules in the path. Following
is puppet's module directory structure.
Mymodule is My module name.
Manifests contains all manifests.
Init.pp is the main class definition file and class definition  must have the same name as module.
Other.pp contains a class mymodule other.pp
Files contains config files to be used by clients.
Facter contains custom facts.
Each manifest in manifests folder should have only one class definition that maps to corresponding file name.
Files in modules use three backslashes to point to file location in files directory.
Templates in modules are used to render output to a file recourse or be used as a variable.

\section{Puppet Facter}

Facter is a tool used by puppet to gather all the environment
variables in a system~\cite{hid-sp18-413-puppet-facter}. Facter comes packaged with puppet agent and
stores the environment variables in key value pair. When agent
requests catalogue from puppet server these facts are available as
variables. Because facts in a operating systems are defined by its
environment varaibles, they are sometimes not enough. Puppets
provides the ability to write custom facts in ruby and these can then
be made available to all puppet clients.

\section{Puppet Heira}

Puppets heira data is used to separate puppet code from configuration
data. Heira enables puppet's code to be reusable.y seperating
configuration data from code Heira enables configuration to be applied
to different environments without change in code. Heira configuration
is defined in heira.yaml,  hierarchical data look up is defined in
heiarchy section and  data dir defines where thesearch  path should
begin. Heira enables puppet code to be more modular and different
configurations can be applied to different systems in a more efficient
way. Newest feature of Heira enables heirarchial configurations for
each environment and module enabling users to change default values
without having to change anything in the actual code.

\section{Puppet DB}
Puppet DB~\cite{hid-sp18-413-puppetdb} is the central database for infrsatructures running pupperdb.  It
stores the most recent facts from all agents, most recent catalogues and reports
from all agents. With catalogues being stored on puppetdb, puppet master is free
to run more catalogues which is specially helpful in large infrastructures. With
the use of puppet DB performance of Puppet infrastructure is enhanced many
folds. Puppet DB serves as a secure storage for all puppet data and comes with a
built in API that can be used to query resources and metrices effectivly.

\section{Puppet Monitoring}
Puppet runs can fail due to multiple reasons~\cite{hid-sp18-413-puppet-monitoring}. Agent failure, network
disruption, server error etc. It becomes necessary to monitor puppet
uns. Puppet logs itslast run status into
lastrunsummary.yaml. RegistrationMetaData plugin for mongodb~\cite{hid-sp18-413-mongodb} and
mcollective~\cite{hid-sp18-413-puppet-mcollective} can be used to store all logs from
agents into a central mongodb database. This is turn can be monitored
and visualized using graphana or other softwares. Puppet has extensive
logs both for servers and agents and another option it to use
elasticsearch, logstatsh and kibana (ELK) stack. Elasticsearch will
perform the indexing and Kibana using Lucene can be used to query logs.

\section{Puppet and Kubernetes}

Kubernetes is Google’s cluster manager and coupled with configuration
management software like puppet it makes managing POD’s, replication
controllers and services easier. Puppet is known for configuring
agents files, resources or services but with kubernetes module it
extends its functionality to easily manage different kubernetes types
like node, event, service, resource quota. Main advantage of using
puppet lies in idempotency i.e if kubernetes is already in desired
state, no changes will be applied. Interface for puppet’s code is
exactly  the same YAML format as
kubernetes~\cite{hid-sp18-413-puppet-kubernetes}. Puppets current
kubernetes module supports configuring pod, service,
replication controller, node, events, endpoint, namespace, secret,
resource quota, limit range and persistant volume.
Puppet’s kubernets module uses kubeclient library to communicate
withkubernetes.''This module includes a configuration tool called kubetool to auto generate all the security parameters, the bootstrap token, and other configurations for your Kubernetes cluster into a Hiera file. The tool is available as a Docker image to simplify installation and use.''~\cite{hid-sp18-413-puppet-kubeclient}
Puppet’s familiar code can be used to configure different kubernetes
resource types.Kubetool creates a YAML file and it should be placed in heira directory. Kubernetes module allows for buisness specific cases
to be implemented. Resource needs to be declared ones and matching
resources can then be spun without describing full state from
beginning. 

\section{Conclusion}

Puppet is a very powerful tool and programming language for configuring and
managing ever changing computer systems. Infrastrucure as code is the main
principle behind puppets ideology. Puppet gives you the ability to run it in
simulation mode and therefore if things are not as desired it is easy to
rollback. Many organisations use puppet for small tasks and as skills enhance i
can be used to manage biggest of infrastructures thus letting users take baby
steps first.Main advantage of puppet lies in the fact that if system is already
in desired state nothing will be changed and deployed. With the fast advent and
advances in cloud computing Puppet is coming to the forefront in helping to
manage and automate vastly huge infrastructures.

\begin{acks}

  The authors would like to thank Dr.~Gregor~von~Laszewski for his support and
  suggestions to write this paper.

\end{acks}

\bibliographystyle{ACM-Reference-Format} \bibliography{report}

